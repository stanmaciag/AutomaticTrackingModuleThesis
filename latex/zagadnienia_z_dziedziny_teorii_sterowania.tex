\chapter{Zagadnienia z dziedziny teorii sterowania występujące w kontekście zadania śledzenia obiektów}
\label{cha:Zagadnienia_z_dziedziny_teorii_sterowania}

\section{Filtr Kalmana}
\label{sec:Filtr_Kalmana}

\textbf{Filtr Kalmana} (ang. \textit{Kalman Filter}, w skórcie \textit{KF}) jest rekurencyjnym algorytmem optymalnej estymacji wektora stanu modelu obiektu, zakładającym jego liniowość oraz rozkład normalny amplitudy występującego szumu procesu i pomiarowego (jest rodzajem optymalnego, rekursywnego filtru Bayesa) \cite{Challa2011}. Od kiedy został po raz pierwszy opublikowany w roku 1960 znalazł szerokie zastosowanie, m.in. w systemach śledzenia obiektów.

Dany jest liniowy i dyskretny układ dynamiczny opisany równaniami \cite{Welch1995}:

\begin{equation}
\label{equ:Filtr_Kalmana_rownanie_procesu}
	\vec{x}_k = \matr{A} \vec{x}_{k-1} + \matr{B} \vec{u}_{k - 1} + \vec{w}_{k-1}
\end{equation}

\begin{equation}
\label{equ:Filtr_Kalmana_rownanie_pomiaru}
	\vec{z}_k = \matr{H} \vec{x}_k + \vec{v}_k
\end{equation}

\noindent
gdzie:

\begin{conditions}
	 k & czas \\
	 \vec{x} & wektor stanu \\
	 \matr{A} & macierz przejścia, wiążąca bieżący stan z poprzednim \\
	 \matr{B} & macierz sterowania, wiążąca bieżący stan z sygnałem sterującym \\
	 \vec{u} & wektor sygnału sterującego \\
	 \vec{w} & wektor szumu procesu \\
	 \vec{z} & wektor pomiaru stanu \\
	 \matr{H} & macierz pomiaru, wiążąca wynik pomiaru ze stanem \\
	 \vec{v} & wektor szumu pomiaru \\
\end{conditions}

Szumy $\vec{w}$ i $\vec{v}$ są od siebie niezależne, mają postać postać szumu białego, a rozkład prawdopodobieństwa ich amplitud ma postać rozkładu normalnego o zerowej wartości oczekiwanej \cite{Welch1995}:

\begin{equation}
\label{equ:Filtr_Kalmana_szum_procesu}
	p(\vec{w}) = N(0, \matr{Q})
\end{equation}

\begin{equation}
\label{equ:Filtr_Kalmana_szum_pomiaru}
	p(\vec{v}) = N(0, \matr{R})
\end{equation}

\noindent
gdzie:

\begin{conditions}
	\matr{Q} & macierz kowariancji szumu procesu \\
	\matr{R} & macierz kowariancji szumu pomiarowego \\
\end{conditions}

Podsumowując, zakłada się pewną zależność pomiędzy bieżącym a poprzednim stanem obiektu, przy czym może być ona zaburzana w nieznany sposób. Stan obiektu nie jest znany, może być jedynie obserwowany za pośrednictwem pomiarów obarczonych pewnym błędem. Celem \textit{KF} jest wyznaczenie estymaty stanu tego obiektu, w sposób statystycznie optymalny ze względu na błąd estymacji.

W algorytmie \textit{KF} rozróżniono estymaty stanu \textit{a priori} $\hat{\vec{x}}_k^-$, wyznaczaną na podstawie poprzedniego stanu, oraz \textit{a posteriori} $\hat{\vec{x}}_k$, uwzględniającą ostatni wynik pomiarów. Błędy estymacji \textit{a priori} i \textit{a posteriori} definiuje się jako \cite{Welch1995}:

\begin{equation}
\label{equ:Filtr_Kalmana_blad_a_priori}
	\vec{e}_k^- \equiv \vec{x}_k - \hat{\vec{x}}_k^-
\end{equation}

\begin{equation}
\label{equ:Filtr_Kalmana_a_posteriori}
	\vec{e}_k \equiv \vec{x}_k - \hat{\vec{x}}_k
\end{equation}

Analogicznie, macierze kowariancji błędów estymacji \textit{a priori} i \textit{a posteriori} mają postaci:

\begin{equation}
\label{equ:Filtr_Kalmana_macierz_kowariancji_a_priori}
	\matr{P}_k^- = E[\vec{e}_k^- {\vec{e}_k^-}^T]
\end{equation}

\begin{equation}
\label{equ:Filtr_Kalmana_macierz_kowariancji_a_posteriori}
	\matr{P}_k = E[\vec{e}_k \vec{e}_k^T]
\end{equation}

\noindent
gdzie:

\begin{conditions}
	E & operator wartości oczekiwanej \\
\end{conditions}

Elementy macierzy kowariancji błędów leżące na ich głównych przekątnych to wariancje błędów estymacji poszczególnych zmiennych stanu, stanowiące miarę niepewności tej estymacji. Pozostałe elementy to kowariancje pomiędzy błędami estymacji poszczególnych zmiennych stanu, określające ich wzajemne powiązanie. Na podstawie estymaty \textit{a priori} można wyznaczyć estymatę \textit{a posteriori}, zgodnie z zależnością \cite{Welch1995}:

\begin{equation}
\label{equ:Filtr_Kalmana_wyznaczenie_estymaty_a_posteriori}
	\hat{\vec{x}}_k = \hat{\vec{x}}_k^- + \matr{K}(\vec{z}_k - \matr{H}\hat{\vec{x}}_k^-)
\end{equation}

\noindent
gdzie:

\begin{conditions}
	K & macierz wzmocnienie Kalmana \\
\end{conditions}

Macierz wzmocnienia Kalmana $\matr{K}$ powinna być zostać określona tak, aby minimalizować wariancję \textit{a posteriori} $\matr{P}_k$, jedna z jej możliwych postaci może zostać obliczona z równania \cite{Welch1995}:

\begin{equation}
\label{equ:Filtr_Kalmana_wzmocnienie_Kalmana}
	\matr{K}_k = \matr{P}_k^- \matr{H}^T ( \matr{H} \matr{P}_k^- \matr{H}^T +\matr{R} ) ^ {-1}
\end{equation}

Z równania \ref{equ:Filtr_Kalmana_wzmocnienie_Kalmana} wynika, że im bardziej elementy macierzy kowariancji błędu pomiarowego $\matr{H}$ zbliżają się do zera, tym bardziej wynik równania \ref{equ:Filtr_Kalmana_wyznaczenie_estymaty_a_posteriori} zależy od wyniku pomiaru $\vec{z}_k$ i jednocześnie mniej zależy od estymaty \textit{a priori} $\hat{\vec{x}}_k^-$. Z drugiej strony, dążenie wartości elementów macierzy kowariancji błędów estymacji \textit{a priori} do zera implikuje spadek znaczenia pomiaru $\vec{z}_k$ i wzrost znaczenia przewidywanego wyniku $\hat{\vec{x}}_k^-$ \cite{Welch1995}.

Algorytm estymacji stanu z wykorzystaniem \textit{KF} składa się dwóch etapów. Pierwszy z nich, określany jako faza predykcji, polega na wyznaczeniu nowej estymaty stanu oraz macierzy kowariancji błędów \textit{a priori} na podstawie ich poprzednich postaci \cite{Welch1995}:

\begin{equation}
\label{equ:Filtr_Kalmana_faza_predykcji_stan}
	\hat{\vec{x}}_k^- = \matr{A} \hat{\vec{x}}_{k-1} + \matr{B} \vec{u}_{k - 1}
\end{equation}

\begin{equation}
\label{equ:Filtr_Kalmana_faza_predykcji_macierz_kowariancji_bledow}
	\matr{P}_k^- = \matr{A} \matr{P}_{k-1} \matr{A}^T + \matr{Q}
\end{equation}

W pierwszy kroku zakłada się początkowe wartości $\hat{\vec{x}}_{k-1}$ oraz $\matr{P}_{k-1}$. Po fazie predykcji następuje faza korekcji, w której uwzględniając wyniki pomiarów wyznacza się wartości \textit{a posteriori} \cite{Welch1995}:

\begin{equation}
\label{equ:Filtr_Kalmana_faza_korekcji_wzmocnienie}
	\matr{K}_k = \matr{P}_k^- \matr{H}^T ( \matr{H} \matr{P}_k^- \matr{H}^T +\matr{R} ) ^ {-1}
\end{equation}

\begin{equation}
\label{equ:Filtr_Kalmana_faza_korekcji_stan}
	\hat{\vec{x}}_k = \hat{\vec{x}}_k^- + \matr{K}_k(\vec{z}_k - \matr{H}\hat{\vec{x}}_k^-)
\end{equation}

\begin{equation}
\label{equ:Filtr_Kalmana_faza_korekcji_macierz_kowariancji_bledow}
	\matr{P}_k = (\matr{I} - \matr{K}_k \matr{H}) \matr{P}_k^-
\end{equation}

Po zakończeniu iteracji wartości \textit{a posteriori} traktowane są jako wartości \textit{a priori} dla kolejnego kroku $k+1$. Macierz kowariancji szumu pomiarowego $\matr{R}$ można zwykle wyznaczyć eksperymentalnie. Bardziej kłopotliwe jest określenie postaci macierzy kowariancji szumu procesu $\matr{Q}$. Przy założeniu dużej dokładności pomiarów przyjęcie dużych wartości elementów $\matr{Q}$, a więc dużej niepewności procesu, może przynieść zadowalające rezultaty. Innym rozwiązaniem jest wykonanie identyfikacji systemu poprzez dobór wartości macierzy $\matr{R}$ i $\matr{Q}$ z wykorzystaniem np. drugiego filtru Kalmana \cite{Welch1995}.

Jak wspomniano wcześniej, zastosowanie \textit{KF} w klasycznej postaci ogranicza się do układów liniowych. W przypadku układów nieliniowych (opisanych równaniami \ref{equ:Rozszerzony_Filtr_Kalmana_rownanie_procesu} i \ref{equ:Rozszerzony_Filtr_Kalmana_rownanie_pomiaru}) znajduje zastosowanie jego zmodyfikowana wersja, nosząca nazwę \textbf{rozszerzonego filtru Kalmana} (ang. \textit{Extended Kalman Filter}, w skrócie \textit{EKF}) \cite{Welch1995}.

\begin{equation}
\label{equ:Rozszerzony_Filtr_Kalmana_rownanie_procesu}
	\vec{x}_k = f(\vec{x}_{k-1}, \vec{u}_{k-1}, \vec{w_{k-1}})
\end{equation}

\begin{equation}
\label{equ:Rozszerzony_Filtr_Kalmana_rownanie_pomiaru}
	\vec{z}_k = h(\vec{x}_k, \vec{v}_k)
\end{equation}

\noindent
gdzie:

\begin{conditions}
	 f & nieliniowa funkcja wiążąca bieżący stan ze stanem poprzednim, sygnałem sterowania oraz szumem procesu \\
	 h & nieliniowa funkcja wiążąca wynik pomiaru ze stanem oraz szumem pomiarowym \\
\end{conditions}

Ponieważ konkretne wartości szumów $\vec{w_k}$ i $\vec{v_k}$ nie są znane, wektor stanu oraz wektor wyników pomiarów można aproksymować przez ich pominięcie \cite{Welch1995}:

\begin{equation}
\label{equ:Rozszerzony_Filtr_Kalmana_rownanie_procesu_aproksymacja}
	\tilde{\vec{x}}_k = f(\hat{\vec{x}}_{k-1}, \vec{u_{k-1}}, 0)
\end{equation}

\begin{equation}
\label{equ:Rozszerzony_Filtr_Kalmana_rownanie_pomiaru_aproksymacja}
	\tilde{\vec{z}}_k = h(\tilde{x}_k, 0)
\end{equation}

\noindent
gdzie:

\begin{conditions}
	 \tilde{\vec{x}}_k & aproksymacja wektora stanu w chwili $k$ \\
	 \tilde{\vec{z}}_k & aproksymacja wyników pomiarów w chwili $k$ \\
	 \hat{\vec{x}}_k & estymata stanu \textit{a posteriori} z poprzedniego kroku $k$
\end{conditions}

Istotną wadą \textit{EKF} jest brak zachowania normalnego rozkładu szumów po poddaniu ich nieliniowym przekształceniom \cite{Welch1995}. W celu dokonania estymacji wykonywana jest linearyzacja równań \ref{equ:Rozszerzony_Filtr_Kalmana_rownanie_pomiaru} i \ref{equ:Rozszerzony_Filtr_Kalmana_rownanie_procesu} poprzez ich rozwinięcie w szereg Taylora w otoczeniu przybliżeń \ref{equ:Rozszerzony_Filtr_Kalmana_rownanie_procesu_aproksymacja} i \ref{equ:Rozszerzony_Filtr_Kalmana_rownanie_pomiaru_aproksymacja} \cite{Welch1995}:

\begin{equation}
\label{equ:Rozszerzony_Filtr_Kalmana_rownanie_procesu_Taylor}
	\vec{x}_k \approx \tilde{\vec{x}}_k + \matr{A}_k(\vec{x}_{k-1} - \hat{\vec{x}}_{k-1}) + \matr{W}_k\vec{w}_{k-1}
\end{equation}

\begin{equation}
\label{equ:Rozszerzony_Filtr_Kalmana_rownanie_pomiaru_Taylor}
	\vec{z}_k \approx \tilde{\vec{z}}_k + \matr{H}_k(\vec{x}_k - \tilde{x}_k) + \matr{V}_k\vec{v}_k
\end{equation}

\noindent
gdzie:

\begin{conditions}
	\matr{A}_k & macierz Jacobiego pochodnych cząstkowych funkcji $f$ ze względu na $\vec{x}$ w chwili $k$ \\
	\matr{W}_k & macierz Jacobiego pochodnych cząstkowych funkcji $f$ ze względu na $\vec{w}$ w chwili $k$ \\
	\matr{H}_k & macierz Jacobiego pochodnych cząstkowych funkcji $h$ ze względu na $\vec{x}$ w chwili $k$ \\
	\matr{V}_k & macierz Jacobiego pochodnych cząstkowych funkcji $h$ ze względu na $\vec{v}$ w chwili $k$ \\
\end{conditions}

\noindent
Elementy macierzy $\matr{A}_k$, $\matr{W}_k$, $\matr{H}_k$ i $\matr{V}_k$ zdefiniowane są następująco:

\begin{equation}
\label{equ:Rozszerzony_Filtr_Kalmana_macierz_A}
	\matr{A}_{k,[i,j]} = \frac{\partial f_{[i]}}{\partial \vec{x}_{[j]}} (\hat{\vec{x}}_{k-1}, \vec{u}_{k-1}, 0)
\end{equation}

\begin{equation}
\label{equ:Rozszerzony_Filtr_Kalmana_macierz_W}
	\matr{W}_{k,[i,j]} = \frac{\partial f_{[i]}}{\partial \vec{w}_{[j]}} (\hat{\vec{x}}_{k-1}, \vec{u}_{k-1}, 0)
\end{equation}

\begin{equation}
\label{equ:Rozszerzony_Filtr_Kalmana_macierz_H}
	\matr{H}_{k,[i,j]} = \frac{\partial h_{[i]}}{\partial \vec{x}_{[j]}} (\tilde{\vec{x}}_k, 0)
\end{equation}

\begin{equation}
\label{equ:Rozszerzony_Filtr_Kalmana_macierz_V}
	\matr{V}_{k,[i,j]} = \frac{\partial h_{[i]}}{\partial \vec{v}_{[j]}} (\tilde{\vec{x}}_k, 0)
\end{equation}

\noindent
Zakłada się definicję błędów przybliżeń wektora stanu $\tilde{\vec{e}}_{\vec{x}_k}$ i pomiarów $\tilde{\vec{e}}_{\vec{z}_k}$:

\begin{equation}
\label{equ:Rozszerzony_Filtr_Kalmana_blad_stanu}
	\tilde{\vec{e}}_{\vec{x}_k} \equiv \vec{x}_k - \tilde{\vec{x}}_k
\end{equation}

\begin{equation}
\label{equ:Rozszerzony_Filtr_Kalmana_blad_pomiaru}
	\tilde{\vec{e}}_{\vec{z}_k} \equiv \vec{z}_k - \tilde{\vec{z}}_k
\end{equation}

Po wstawieniu równań \ref{equ:Rozszerzony_Filtr_Kalmana_blad_stanu} i \ref{equ:Rozszerzony_Filtr_Kalmana_blad_pomiaru} do równań \ref{equ:Rozszerzony_Filtr_Kalmana_rownanie_procesu_Taylor} i \ref{equ:Rozszerzony_Filtr_Kalmana_rownanie_pomiaru_Taylor} przekształceniach \cite{Welch1995}:

\begin{equation}
\label{equ:Rozszerzony_Filtr_Kalmana_blad_stanu_cd}
	\tilde{\vec{e}}_{\vec{x}_k} \approx \matr{A}_k(\vec{x}_{k-1} - \hat{\vec{x}}_{k-1}) + \vec{\epsilon}_k
\end{equation}

\begin{equation}
\label{equ:Rozszerzony_Filtr_Kalmana_blad_pomiaru_cd}
	\tilde{\vec{e}}_{\vec{z}_k} \approx \matr{H}_k \tilde{\vec{e}}_{\vec{x}_k} + \vec{\eta}_k
\end{equation}

\noindent
gdzie:

\begin{conditions}
	\vec{\epsilon}_k & niezależna zmienna losowa o zerowej wartości średniej i macierzy kowariancji $\matr{W}\matr{Q}_k\matr{W}^T$ \\
	\vec{\eta}_k & niezależna zmienna losowa o zerowej wartości średniej i macierzy kowariancji $\matr{V}\matr{R}_k\matr{V}^T$ \\
\end{conditions}

Równania \ref{equ:Rozszerzony_Filtr_Kalmana_blad_stanu_cd} i \ref{equ:Rozszerzony_Filtr_Kalmana_blad_pomiaru_cd} są liniowe i mają podobną postać jak równania procesu \ref{equ:Filtr_Kalmana_rownanie_procesu} i pomiaru \ref{equ:Filtr_Kalmana_rownanie_pomiaru} \textit{KF}, co sugeruje wykorzystanie drugiego, hipotetycznego filtru Kalmana do estymacji błędu predykcji $\tilde{\vec{e}}_{\vec{x}_k}$ \cite{Welch1995}. Jego estymacja, oznaczona jako $\hat{\vec{e}}_k$, może następnie służyć do wyznaczenia estymaty stanu \textit{a posteriori} układu nieliniowego, na podstawie równania \ref{equ:Rozszerzony_Filtr_Kalmana_blad_stanu}:

\begin{equation}
\label{equ:Rozszerzony_Filtr_Kalmana_estymacja_stanu}
	\hat{\vec{x}}_k = \tilde{\vec{x}}_k + \hat{\vec{e}}_k
\end{equation}

Zmienne losowe $\vec{\epsilon}_k$ i $\vec{\eta}_k$ mają w przybliżeniu rozkłady prawdopodobieństwa odpowiednio $p(\vec{\epsilon}_k) \sim N(0, \matr{W}\matr{Q}_k\matr{W}^T)$ i $p(\vec{\eta}_k) \sim N(0, \matr{V}\matr{R}_k\matr{V}^T)$. Korzystając z równania \ref{equ:Filtr_Kalmana_wyznaczenie_estymaty_a_posteriori}, oraz zakładając zerową wartość predykcji $\hat{\vec{e}}_k$ \cite{Welch1995}:

\begin{equation}
\label{equ:Rozszerzony_Filtr_Kalmana_estymacja_bledu}
	\hat{\vec{e}}_k = \matr{K}_k \tilde{\vec{e}}_{\vec{z}_k}
\end{equation}

Wstawienie równania \ref{equ:Rozszerzony_Filtr_Kalmana_estymacja_bledu} do \ref{equ:Rozszerzony_Filtr_Kalmana_estymacja_stanu} oraz podstawienie aproksymowanej wartości błędu pomiaru na postawie równania \ref{equ:Rozszerzony_Filtr_Kalmana_blad_pomiaru} ujawnia brak konieczności stosowania drugiego \textit{KF} i pozwala na wyznaczenie postaci równania korekcji stanu \textit{EKF} \cite{Welch1995}:

\begin{equation}
\label{equ:Rozszerzony_Filtr_Kalmana_estymacja_stanu_cd}
	\hat{\vec{x}}_k = \tilde{\vec{x}}_k + \matr{K}_k (\vec{z}_k - \tilde{\vec{z}}_k)
\end{equation}

Macierz wzmocnienia Kalmana $\matr{K}_k$ wyznacza się analogicznie jak w \textit{KF}, na podstawie równania \ref{equ:Filtr_Kalmana_faza_korekcji_macierz_kowariancji_bledow}, podstawiając zmodyfikowaną postać macierzy kowariancji błędów pomiaru.

Równania \textit{EKF} fazy predykcji przyjmują następującą postać \cite{Welch1995}:

\begin{equation}
\label{equ:Rozszerzony_Filtr_Kalmana_faza_predykcji_stan}
	\hat{\vec{x}}_k^- = f(\vec{x}_{k-1}, \vec{u}_{k-1}, 0)
\end{equation}

\begin{equation}
\label{equ:Rozszerzony_Filtr_Kalmana_faza_predykcji_macierz_kowariancji_bledow}
	\matr{P}_k^- = \matr{A}_k \matr{P}_{k-1} \matr{A}_k^T + \matr{W}_k\matr{Q}_{k-1}\matr{W}_k^T
\end{equation}

Równania fazy korekcji \cite{Welch1995}:

\begin{equation}
\label{equ:Rozszerzony_Filtr_Kalmana_faza_korekcji_wzmocnienie}
	\matr{K}_k = \matr{P}_k^- \matr{H}_k^T ( \matr{H}_k \matr{P}_k^- \matr{H}_k^T + \matr{V}_k\matr{R}_k\matr{V}_k^T ) ^ {-1}
\end{equation}

\begin{equation}
\label{equ:Rozszerzony_Filtr_Kalmana_faza_korekcji_stan}
	\hat{\vec{x}}_k = \hat{\vec{x}}_k^- + \matr{K}_k(\vec{z}_k - h(\hat{\vec{x}}_k^-,0))
\end{equation}

\begin{equation}
\label{equ:Rozszerzony_Filtr_Kalmana_faza_korekcji_macierz_kowariancji_bledow}
	\matr{P}_k = (\matr{I} - \matr{K}_k \matr{H}_k) \matr{P}_k^-
\end{equation}