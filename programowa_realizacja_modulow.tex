\chapter{Programowa realizacja modułów śledzenia obiektów}
\label{cha:Programowa_realizacja_modulow_sledzenia_obiektow}

Dalsza część niniejszej pracy dotyczy praktycznej realizacji modułów wizyjnego śledzenia obiektów. W kolejnych rozdziałach przedstawiono algorytm nadrzędny, definiujący ogólny schemat ich działania oraz omówiono sposoby implementacji jego poszczególnych kroków.

\section{Założenia projektowe}
\label{sec:Zalozenia_projektowe}

Ze względu na obszar aplikacji, tj. robotykę mobilną, moduł powinien wykazywać następujące cechy:
\begin{itemize}

	\item Niska złożoność obliczeniowa - wymóg praktyczny, podyktowany relatywnie niską mocą obliczeniową procesorów pokładowych, stanowiącą skutek wymogu niewielkiego poboru mocy
	\item Wysoka odporność na zmiany iluminacji oraz przesłonięcia śledzonego obiektu - niezbędna cecha modułu umożliwiająca jego poprawne działanie w przypadku założenia dynamicznego charakteru obserwowanej sceny
	\item Wysoka odporność na szum, \textit{motion blur}, oraz inne niekorzystne zjawiska będące skutkiem gorszą jakością sensora wizyjnego
	\item Zdolność do działania w czasie rzeczywistym, rozumiana jako zdolność do wykonania wszystkich obliczeń pomiędzy dwoma kolejnymi klatkami rejestrowanymi przez sensor wizyjny - niższa częstotliwość próbkowania (omijanie klatek) skutkuje zwiększeniem różnic pomiędzy klatkami przetwarzanymi przez moduł, co bezpośrednio przekłada się na pogorszenie jakości procesu śledzenia 
	
\end{itemize}

Moduł powinien dostarczać następujących funkcjonalności:

\begin{itemize}

	\item Zdefiniowanie obiektu zainteresowania na podstawie zawierającego go obszaru zainteresowania na klatce początkowej
	\item Estymacja położenia obiektu na kolejnych klatkach
	\item Estymacja orientacji (rozumianej jako obrotu wokół osi prostopadłej do płaszczyzny obrazu) obiektu
	\item Estymacja wymiarów obiektu w postaci wymiarów ograniczającego go prostokąta
	\item Pomiar prawdopodobieństwa powodzenia procesu śledzenia na podstawie estymowanego stanu
	\item Określenie niepowodzenia zadania śledzenia w przypadku utraty informacji o stanie obiektu zainteresowania

\end{itemize}

\section{Algorytm nadrzędny}
\label{sec:Algorytm_nadrzedny}

Nadrzędny algorytm realizowany przez moduł śledzenia składa się z trzech zasadniczych etapów, z których dwa ostatnie wykonywane są iteracyjnie:

\begin{enumerate}

	\item \textbf{Zogniskowanie} - etap początkowy, dane wejściowe w postaci zewnętrznie zadanego obszaru zainteresowania, dane wyjściowe w postaci modelu obiektu zainteresowania
	\item \textbf{Śledzenie} - etap zasadniczy, dane wejściowe w postaci poprzedniego położenia obiektu zainteresowania (oraz dodatkowo jego orientacji i wymiarów) oraz jego modelu, dane wyjściowe w postaci estymacji kolejnego położenia obiektu zainteresowania oraz procentowego współczynnika prawdopodobieństwa powodzenia procesu śledzenia
	\item \textbf{Adaptacja} - etap uzupełniający, dane wejściowe w postaci modelu obiektu zainteresowania, dane wyjściowe w postaci modelu uaktualnionego o informację wniesioną przez ostatnią klatkę

\end{enumerate}

\section{Wybór algorytmów zasadniczych}
\label{sec:Wybor_algorytmow_zasadniczych}

W ramach przeprowadzonego w we wcześniejszej części niniejszej pracy przeglądu literaturowego wyodrębniono i opisano podstawowe algorytmy śledzenia wizyjnego stosowane w robotyce mobilnej, tj.:

\begin{enumerate}

	\item Algorytm Lucasa-Kanade'a
	\item Algorytm MeanShift
	\item Algorytm CAMShift
	\item Filtr Kalmana

\end{enumerate}

Implementacja osobnego modułu dla każdego wymienionego wyżej algorytmu jest zadaniem rozmiarem przekraczającym zakres niniejszej pracy, w związku z tym z dalszych rozważań i realizacji wykluczono algorytmy gorzej wpisujące się w przedstawiony wyżej projekt. W pierwszej kolejności wykluczono algorytm \textit{MeanShift}, ze względu na jego nieprzystosowanie do zmiany skali śledzonego obiektu oraz mniej stabilną złożoność obliczeniową w porównaniu do jego rozwinięcia, czyli algorytmu \textit{CAMShift}. Następnie zrezygnowano z wdrożenia modułu opartego o filtr Kalmana, który w zadaniu śledzenia obiektów w zasadzie rolę algorytmu pomocniczego (zmniejszając szum, zawężając obszar poszukiwań, etc.). Moduł wykorzystujący filtr Kalmana musiałby posiadać funkcję pomiaru stanu obiektu, polegającą \textit{de facto} na wykryciu obiektu zainteresowania. Wykrywanie obiektów nie stanowi przedmiotu rozważań niniejszej pracy, dodatkowo jest ogólnie rzecz biorąc dużo bardziej złożone obliczeniowo, co można zauważyć np. poprzez porównanie złożoności często wykorzystywanej w nim metody \textit{SIFT} (rozdział \ref{subsec:Punkty_charakterystyczne}) ze złożonością algorytmu wyznaczania punktów zdatnych do śledzenia (rozdział \ref{subsec:Algorytm_Kanade_Lucasa_Tomasi}).